% Warning: LuaLaTeX or XeLaTeX required
\documentclass{article}
\usepackage[margin=1in]{geometry}
\usepackage{hyperref}
\hypersetup{
    colorlinks,
    citecolor=black,
    filecolor=black,
    linkcolor=black,
    urlcolor=black
}

\makeatletter
 \renewcommand*\l@subsection{\@dottedtocline{2}{1.8em}{3.2em}}
\makeatother

\usepackage{fontspec}
\setmonofont{Unifont}

\usepackage{xcolor}
\usepackage{mathtools}

\title{\u{c} Documentation}
\author{Louis A. Burke (lburke@labprogramming.net)}

\newenvironment{operator}[2]{%
    \newcommand{\applymapto}{\ensuremath{\xmapsto{\texttt{\detokenize{#2}}}}}
    \subsection{#1}
    The \textit{#1} operator is represented by \colorbox{gray!30} {\texttt{\detokenize{#2}}}.
    \begin{enumerate}}{\end{enumerate}}
\providecommand{\op}[2]{\item\texttt{(\detokenize{#1})} $\applymapto $
\texttt{(\detokenize{#2})}}

\newenvironment{verb_operator}[2]{%
    \newcommand{\applymapto}{\ensuremath{\xmapsto{\texttt{#2}}}}
    \subsection{#1}
    The \textit{#1} operator is represented by \colorbox{gray!30} {\texttt{#2}}.
    \begin{enumerate}}{\end{enumerate}}
\providecommand{\op}[2]{\item\texttt{(\detokenize{#1})} $\applymapto $
\texttt{(\detokenize{#2})}}

\begin{document}
\maketitle

\tableofcontents
\clearpage

\section{Introduction}

\u{c} is a stack-based tacid code golf language. What sets it apart from many
other golfing languages is that it cares deeply about the type of data being
passed around.

By 'standard' the language performs extremely differently depending on the types
of the input. However, since there is no common standard way to input a number
as a natural number instead of an integer, for example, we specify our input
types at compile-time. In theory the compiler could also produce sources for
each possible combination of inputs, but this would be costly. The point is that
while the compiler has to specify input types, this does not count as part of
the source, but rather a deficiency of the compiler. An ideal \u{c} environment
would have some sort of drop-down menu next to each input and would run the
program on exactly the input types given.

For example consider the 1-byte program \verb|ℙ|. If the input is a single
natural number, this is the primality checking program. If there is no input,
this is instead the program which prints a list of all prime numbers.

\section{Code Page}

The following is the code page for \u{c}:

\begin{tabular}{c c c c c c c c c c c c c c c c c}
\verb| | & \verb|0| & \verb|1| & \verb|2| & \verb|3| & \verb|4| & \verb|5| &
\verb|6| & \verb|7| & \verb|8| & \verb|9| & \verb|A| & \verb|B| & \verb|C| &
\verb|D| & \verb|E| & \verb|F| \\
\verb|0| & \verb|∅| & \verb|¡| & \verb|±| & \verb|°| & \verb|¬| & \verb|¢| &
\verb|£| & \verb|¤| & \verb|¥| & \verb|¦| & \verb|¶| & \verb|«| & \verb|»| &
\verb|₪| & \verb|ℶ| & \verb|♠| \\
\verb|1| & \verb|Þ| & \verb|Ð| & \verb|ß| & \verb|Ƿ| & \verb|Ȝ| & \verb|ſ| &
\verb|µ| & \verb|⅟| & \verb|↶| & \verb|↷| & \verb|↹| & \verb|¿| & \verb|⚅| &
\verb|♭| & \verb|♯| & \verb|⍏| \\
\verb|2| & \verb| | & \verb|!| & \verb|"| & \verb|#| & \verb|$| & \verb|%| &
\verb|&| & \verb|'| & \verb|(| & \verb|)| & \verb|*| & \verb|+| & \verb|,| &
\verb|-| & \verb|.| & \verb|/| \\
\verb|3| & \verb|0| & \verb|1| & \verb|2| & \verb|3| & \verb|4| & \verb|5| &
\verb|6| & \verb|7| & \verb|8| & \verb|9| & \verb|:| & \verb|;| & \verb|<| &
\verb|=| & \verb|>| & \verb|?| \\
\verb|4| & \verb|@| & \verb|A| & \verb|B| & \verb|C| & \verb|D| & \verb|E| &
\verb|F| & \verb|G| & \verb|H| & \verb|I| & \verb|J| & \verb|K| & \verb|L| &
\verb|M| & \verb|N| & \verb|O| \\
\verb|5| & \verb|P| & \verb|Q| & \verb|R| & \verb|S| & \verb|T| & \verb|U| &
\verb|V| & \verb|W| & \verb|X| & \verb|Y| & \verb|Z| & \verb|[| & \verb|\| &
\verb|]| & \verb|^| & \verb|_| \\
\verb|6| & \verb|`| & \verb|a| & \verb|b| & \verb|c| & \verb|d| & \verb|e| &
\verb|f| & \verb|g| & \verb|h| & \verb|i| & \verb|j| & \verb|k| & \verb|l| &
\verb|m| & \verb|n| & \verb|o| \\
\verb|7| & \verb|p| & \verb|q| & \verb|r| & \verb|s| & \verb|t| & \verb|u| &
\verb|v| & \verb|w| & \verb|x| & \verb|y| & \verb|z| & \verb|{| & \verb/|/ &
\verb|}| & \verb|~| & \verb|÷| \\
\verb|8| & \verb|∀| & \verb|∃| & \verb|∄| & \verb|∆| & \verb|∇| & \verb|∈| &
\verb|∉| & \verb|∎| & \verb|∏| & \verb|∑| & \verb|√| & \verb|∞| & \verb|∧| &
\verb|∨| & \verb|∩| & \verb|∪| \\
\verb|9| & \verb|∫| & \verb|∴| & \verb|∵| & \verb|≝| & \verb|≠| & \verb|≡| &
\verb|≤| & \verb|≥| & \verb|≰| & \verb|≱| & \verb|⊆| & \verb|⊇| & \verb|⊈| &
\verb|⊉| & \verb|⋈| & \verb|∝| \\
\verb|A| & \verb|⁰| & \verb|¹| & \verb|²| & \verb|³| & \verb|⁴| & \verb|⁵| &
\verb|⁶| & \verb|⁷| & \verb|⁸| & \verb|⁹| & \verb|⁻| & \verb|←| & \verb|↑| &
\verb|→| & \verb|↓| & \verb|↺| \\
\verb|B| & \verb|₀| & \verb|₁| & \verb|₂| & \verb|₃| & \verb|₄| & \verb|₅| &
\verb|₆| & \verb|₇| & \verb|₈| & \verb|₉| & \verb|□| & \verb|◇| & \verb|⟠| &
\verb|⊤| & \verb|⊥| & \verb|⊦| \\
\verb|C| & \verb|½| & \verb|⌫| & \verb|Ɂ| & \verb|‼| & \verb|†| & \verb|‡| &
\verb|•| & \verb|‘| & \verb|’| & \verb|“| & \verb|”| & \verb|⋯| & \verb|⋮| &
\verb|⋰| & \verb|⋱| & \verb|⁜| \\
\verb|D| & \verb|Ⅰ| & \verb|Ⅱ| & \verb|Ⅲ| & \verb|Ⅳ| & \verb|Ⅴ| & \verb|Ⅵ| &
\verb|Ⅶ| & \verb|Ⅷ| & \verb|Ⅸ| & \verb|Ⅹ| & \verb|Ⅺ| & \verb|Ⅻ| & \verb|Ⅼ| &
\verb|Ⅽ| & \verb|Ⅾ| & \verb|Ⅿ| \\
\verb|E| & \verb|ɑ| & \verb|β| & \verb|ↁ| & \verb|ↂ| & \verb|ↇ| & \verb|ↈ| &
\verb|↤| & \verb|⇤| & \verb|δ| & \verb|Γ| & \verb|Θ| & \verb|Λ| & \verb|Ξ| &
\verb|Φ| & \verb|Ψ| & \verb|Ω| \\
\verb|F| & \verb|ℂ| & \verb|ℍ| & \verb|ℕ| & \verb|ℙ| & \verb|ℚ| & \verb|ℝ| &
\verb|ℤ| & \verb|π| & \verb|ℯ| & \verb|א| & \verb|℞| & \verb|⊕| & \verb|⊖| &
\verb|⊗| & \verb|⊘| & \verb|⊜| \\
\end{tabular}
% This cleans up vim highlighting... $

\section{Execution}

%TODO: maybe more tacit-style. Each function is chained and consumes/produces
%arg(s).

Execution of \u{c} occurs by first creating an implicit stack (not to be
confused with a stack datatype) of all inputs. Then each operation in the source
is applied to the top item of the stack from left to right. Thus you can picture
the definition of \u{c} as a list of 256 operations on each datatype it
supports. Many of these may consume elements of the implicit stack or else
change the data on the stack in various ways. Manipulating the correct datatypes
is the key to a short \u{c} program.

\section{Types} % TODO: Move the operations into this section

\subsection{Numbers}

These are the numeric types available in \u{c}:
\begin{enumerate}
    \item \texttt{integer} - Integers
    \item \texttt{number} - Real numbers
    \item \texttt{rational} - Rational numbers
    \item \texttt{complex} - Complex numbers
    \item \texttt{natural} - Natural numbers
    \item \texttt{quaternion} - Quaternion numbers
\end{enumerate}

\subsection{Mathematical Objects}

These are the mathematical types available in \u{c}:
\begin{enumerate}
    \item \texttt{set} - Sets
    \item \texttt{graph} - Graphs
    \item \texttt{digraph} - Digraphs
    \item \texttt{mgraph} - Multigraphs
    \item \texttt{group} - Groups
    \item \texttt{vector} - Vectors
    \item \texttt{matrix} - Matrices
\end{enumerate}

\subsection{Computer Science Objects}

These are the computer science types available in \u{c}:
\begin{enumerate}
    \item \texttt{stack} - Stacks
    \item \texttt{queue} - Queues
    \item \texttt{string} - Strings
    \item \texttt{list} - Lists
    \item \texttt{tuple} - Tuples
    \item \texttt{stream} - Streams
\end{enumerate}

\subsection{Games}

These are the game types available in \u{c}:
\begin{enumerate}
    \item \texttt{poker} - Poker cards/hands
    \item \texttt{bridge} - Bridge cards/hands
\end{enumerate}

\subsection{Miscellaneous}

These are the miscellaneous types available in \u{c}:
\begin{enumerate}
    \item \texttt{image} - Images
    \item \texttt{unit} - Standard Units
    \item \texttt{grid} - Grids of things
\end{enumerate}

\section{Operators}

Each operator is defined by a list of actions it takes on a specific argument.
The argument is specified by a tuple to tuple type specification, then a short
description of what the operator returns.

The actual argument tuple could be longer than that listed, in which case the
function merely effects the first arguments which it needs.

Each operator will go through each valid mapping it has in order to decide what
to do. Thus precedence is encoded into these listings. Some types which
construct literals are not listed. Instead these types are presented in ALL CAPS
to indicate that every operator has intrinsically defined actions on those
types.

The special types \texttt{*} and \texttt{\_} indicate anything and a single
element respectively.

\begin{operator}{Null}{∅}
    \op{}{set}: Returns the null set.
\end{operator}

\begin{operator}{Top}{¡}
    \op{_,*}{_}: Returns the first argument.
\end{operator}

\begin{operator}{Plus Minus}{±}
    \op{number}{{number}}: Returns a set containing both positive and negative versions of the number.
\end{operator}

\begin{operator}{Degrees}{°}
    \op{number}{number}: Convert from degrees to radians.
\end{operator}

\begin{operator}{Not}{¬}
    \op{_}{bool}: Logical not.
\end{operator}

\begin{operator}{Cent}{¢}
\item.
\end{operator}

\begin{operator}{Pound}{£}
\item.
\end{operator}

\begin{operator}{Graph}{¤}
\item.
\end{operator}

\begin{operator}{Split}{¥}
\item.
\end{operator}

\begin{operator}{Slit}{¦}
\item.
\end{operator}

\begin{operator}{Drop}{¶}
\item.
\end{operator}

\begin{operator}{Shift Left}{«}
\item.
\end{operator}

\begin{operator}{Shift Right}{»}
\item.
\end{operator}

\begin{operator}{Convolve}{₪}
\item.
\end{operator}

\begin{operator}{Bet}{ℶ}
\item.
\end{operator}

\begin{operator}{Spade}{♠}
\item.
\end{operator}

\begin{operator}{Bump}{Þ}
\item.
\end{operator}

\begin{operator}{Enter}{Ð}
\item.
\end{operator}

\begin{operator}{Sharp S}{ß}
\item.
\end{operator}

\begin{operator}{Win}{Ƿ}
\item.
\end{operator}

\begin{operator}{Yogh}{Ȝ}
\item.
\end{operator}

\begin{operator}{Not End}{ſ}
\item.
\end{operator}

\begin{operator}{Mu}{µ}
\item.
\end{operator}

\begin{operator}{Reciprocal}{⅟}
\item.
\end{operator}

\begin{operator}{Back}{↶}
\item.
\end{operator}

\begin{operator}{Forward}{↷}
\item.
\end{operator}

\begin{operator}{Back and Forth}{↹}
\item.
\end{operator}

\begin{operator}{Irony}{¿}
\item.
\end{operator}

\begin{operator}{Die}{⚅}
\item.
\end{operator}

\begin{operator}{Flat}{♭}
\item.
\end{operator}

\begin{operator}{Sharp}{♯}
\item.
\end{operator}

\begin{operator}{Through}{⍏}
\item.
\end{operator}

\begin{operator}{Space}{ }
\item.
\end{operator}

\begin{operator}{Bang}{!}
\item.
\end{operator}

\begin{operator}{Quote}{"}
\item.
\end{operator}

\begin{verb_operator}{Number}{\#}
\item.
\end{verb_operator}

\begin{operator}{Money}{$} % To appease vim again $
\item.
\end{operator}

\begin{verb_operator}{Percent}{\%}
\item.
\end{verb_operator}

\begin{operator}{And}{&}
\item.
\end{operator}

\begin{operator}{Apostrophe}{'}
\item.
\end{operator}

\begin{operator}{Open}{(}
\item.
\end{operator}

\begin{operator}{Close}{)}
\item.
\end{operator}

\begin{operator}{Star}{*}
\item.
\end{operator}

\begin{operator}{Plus}{+}
\item.
\end{operator}

\begin{operator}{Comma}{,}
\item.
\end{operator}

\begin{operator}{Minus}{-}
\item.
\end{operator}

\begin{operator}{Dot}{.}
\item.
\end{operator}

\begin{operator}{Slash}{/}
\item.
\end{operator}

\begin{operator}{Zero}{0}
\item.
\end{operator}

\begin{operator}{One}{1}
\item.
\end{operator}

\begin{operator}{Two}{2}
\item.
\end{operator}

\begin{operator}{Three}{3}
\item.
\end{operator}

\begin{operator}{Four}{4}
\item.
\end{operator}

\begin{operator}{Five}{5}
\item.
\end{operator}

\begin{operator}{Six}{6}
\item.
\end{operator}

\begin{operator}{Seven}{7}
\item.
\end{operator}

\begin{operator}{Eight}{8}
\item.
\end{operator}

\begin{operator}{Nine}{9}
\item.
\end{operator}

\begin{operator}{Colon}{:}
\item.
\end{operator}

\begin{operator}{Semi-colon}{;}
\item.
\end{operator}

\begin{operator}{Less}{<}
\item.
\end{operator}

\begin{operator}{Equals}{=}
    \op{list,list}{list}: List with true wherever the elements in both lists are the same.
    \op{number,number}{bool}: True if the two numbers are equal.
\end{operator}

\begin{operator}{Greater}{>}
\item.
\end{operator}

\begin{operator}{Query}{?}
\item.
\end{operator}

\begin{operator}{At}{@}
\item.
\end{operator}

\begin{operator}{A}{A}
\item.
\end{operator}

\begin{operator}{B}{B}
\item.
\end{operator}

\begin{operator}{C}{C}
\item.
\end{operator}

\begin{operator}{D}{D}
\item.
\end{operator}

\begin{operator}{E}{E}
\item.
\end{operator}

\begin{operator}{F}{F}
\item.
\end{operator}

\begin{operator}{G}{G}
\item.
\end{operator}

\begin{operator}{H}{H}
\item.
\end{operator}

\begin{operator}{I}{I}
\item.
\end{operator}

\begin{operator}{J}{J}
\item.
\end{operator}

\begin{operator}{K}{K}
\item.
\end{operator}

\begin{operator}{L}{L}
\item.
\end{operator}

\begin{operator}{M}{M}
\item.
\end{operator}

\begin{operator}{N}{N}
\item.
\end{operator}

\begin{operator}{O}{O}
\item.
\end{operator}

\begin{operator}{P}{P}
\item.
\end{operator}

\begin{operator}{Q}{Q}
\item.
\end{operator}

\begin{operator}{R}{R}
\item.
\end{operator}

\begin{operator}{S}{S}
\item.
\end{operator}

\begin{operator}{T}{T}
\item.
\end{operator}

\begin{operator}{U}{U}
\item.
\end{operator}

\begin{operator}{V}{V}
\item.
\end{operator}

\begin{operator}{W}{W}
\item.
\end{operator}

\begin{operator}{X}{X}
\item.
\end{operator}

\begin{operator}{Y}{Y}
\item.
\end{operator}

\begin{operator}{Z}{Z}
\item.
\end{operator}

\begin{operator}{Start}{[}
\item.
\end{operator}

\begin{verb_operator}{Backslash}{\textbackslash}
\item.
\end{verb_operator}

\begin{operator}{End}{]}
\item.
\end{operator}

\begin{verb_operator}{Up}{\^}
\item.
\end{verb_operator}

\begin{verb_operator}{Under}{\_}
\item.
\end{verb_operator}

\begin{operator}{Tick}{`}
\item.
\end{operator}

\begin{operator}{a}{a}
\item.
\end{operator}

\begin{operator}{b}{b}
\item.
\end{operator}

\begin{operator}{c}{c}
\item.
\end{operator}

\begin{operator}{d}{d}
\item.
\end{operator}

\begin{operator}{e}{e}
\item.
\end{operator}

\begin{operator}{f}{f}
\item.
\end{operator}

\begin{operator}{g}{g}
\item.
\end{operator}

\begin{operator}{h}{h}
\item.
\end{operator}

\begin{operator}{i}{i}
\item.
\end{operator}

\begin{operator}{j}{j}
\item.
\end{operator}

\begin{operator}{k}{k}
\item.
\end{operator}

\begin{operator}{l}{l}
\item.
\end{operator}

\begin{operator}{m}{m}
\item.
\end{operator}

\begin{operator}{n}{n}
\item.
\end{operator}

\begin{operator}{o}{o}
\item.
\end{operator}

\begin{operator}{p}{p}
\item.
\end{operator}

\begin{operator}{q}{q}
\item.
\end{operator}

\begin{operator}{r}{r}
\item.
\end{operator}

\begin{operator}{s}{s}
\item.
\end{operator}

\begin{operator}{t}{t}
\item.
\end{operator}

\begin{operator}{u}{u}
\item.
\end{operator}

\begin{operator}{v}{v}
\item.
\end{operator}

\begin{operator}{w}{w}
\item.
\end{operator}

\begin{operator}{x}{x}
\item.
\end{operator}

\begin{operator}{y}{y}
\item.
\end{operator}

\begin{operator}{z}{z}
\item.
\end{operator}

\begin{verb_operator}{Begin}{\{}
\item.
\end{verb_operator}

\begin{operator}{Pipe}{|}
\item.
\end{operator}

\begin{verb_operator}{End}{\}}
\item.
\end{verb_operator}

\begin{operator}{Tilde}{~}
\item.
\end{operator}

\begin{operator}{Divide}{÷}
\item.
\end{operator}

\begin{operator}{Universal}{∀}
\item.
\end{operator}

\begin{operator}{Existential}{∃}
\item.
\end{operator}

\begin{operator}{Non Existential}{∄}
\item.
\end{operator}

\begin{operator}{Increment}{∆}
\item.
\end{operator}

\begin{operator}{Decrement}{∇}
\item.
\end{operator}

\begin{operator}{In}{∈}
    \op{_,set}{_,set}: Coerces the top element to an element of the set.
\end{operator}

\begin{operator}{Not In}{∉}
\item.
\end{operator}

\begin{operator}{Box}{∎}
\item.
\end{operator}

\begin{operator}{Product}{∏}
\item.
\end{operator}

\begin{operator}{Sum}{∑}
\item.
\end{operator}

\begin{operator}{Root}{√}
\item.
\end{operator}

\begin{operator}{Infinity}{∞}
\item.
\end{operator}

\begin{operator}{Conjunction}{∧}
\item.
\end{operator}

\begin{operator}{Disjunction}{∨}
\item.
\end{operator}

\begin{operator}{Intersection}{∩}
\item.
\end{operator}

\begin{operator}{Union}{∪}
\item.
\end{operator}

\begin{operator}{Integral}{∫}
\item.
\end{operator}

\begin{operator}{Therefore}{∴}
\item.
\end{operator}

\begin{operator}{Because}{∵}
\item.
\end{operator}

\begin{operator}{Define}{≝}
\item.
\end{operator}

\begin{operator}{Not Equals}{≠}
\item.
\end{operator}

\begin{operator}{Exact Equals}{≡}
    \op{_,_}{bool}: True if the two elements on top of the stack are identical, false otherwise.
\end{operator}

\begin{operator}{Not Greater}{≤}
\item.
\end{operator}

\begin{operator}{Not Less}{≥}
\item.
\end{operator}

\begin{operator}{Make Greater}{≰}
\item.
\end{operator}

\begin{operator}{Make Less}{≱}
\item.
\end{operator}

\begin{operator}{Subset}{⊆}
\item.
\end{operator}

\begin{operator}{Superset}{⊇}
\item.
\end{operator}

\begin{operator}{Make Superset}{⊈}
\item.
\end{operator}

\begin{operator}{Make Subset}{⊉}
\item.
\end{operator}

\begin{operator}{Join}{⋈}
\item.
\end{operator}

\begin{operator}{Proportional}{∝}
\item.
\end{operator}

\begin{operator}{Super Zero}{⁰}
\item.
\end{operator}

\begin{operator}{Super One}{¹}
\item.
\end{operator}

\begin{operator}{Super Two}{²}
\item.
\end{operator}

\begin{operator}{Super Three}{³}
\item.
\end{operator}

\begin{operator}{Super Four}{⁴}
\item.
\end{operator}

\begin{operator}{Super Five}{⁵}
\item.
\end{operator}

\begin{operator}{Super Six}{⁶}
\item.
\end{operator}

\begin{operator}{Super Seven}{⁷}
\item.
\end{operator}

\begin{operator}{Super Eight}{⁸}
\item.
\end{operator}

\begin{operator}{Super Nine}{⁹}
\item.
\end{operator}

\begin{operator}{Super Minus}{⁻}
\item.
\end{operator}

\begin{operator}{Left}{←}
\item.
\end{operator}

\begin{operator}{Up}{↑}
\item.
\end{operator}

\begin{operator}{Right}{→}
\item.
\end{operator}

\begin{operator}{Down}{↓}
\item.
\end{operator}

\begin{operator}{Loop}{↺}
\item.
\end{operator}

\begin{operator}{Sub 0}{₀}
\item.
\end{operator}

\begin{operator}{Sub 1}{₁}
\item.
\end{operator}

\begin{operator}{Sub 2}{₂}
\item.
\end{operator}

\begin{operator}{Sub 3}{₃}
\item.
\end{operator}

\begin{operator}{Sub 4}{₄}
\item.
\end{operator}

\begin{operator}{Sub 5}{₅}
\item.
\end{operator}

\begin{operator}{Sub 6}{₆}
\item.
\end{operator}

\begin{operator}{Sub 7}{₇}
\item.
\end{operator}

\begin{operator}{Sub 8}{₈}
\item.
\end{operator}

\begin{operator}{Sub 9}{₉}
\item.
\end{operator}

\begin{operator}{Necessarily}{□}
\item.
\end{operator}

\begin{operator}{Possibly}{◇}
\item.
\end{operator}

\begin{operator}{Impossibly}{⟠}
\item.
\end{operator}

\begin{operator}{True}{⊤}
\item.
\end{operator}

\begin{operator}{False}{⊥}
\item.
\end{operator}

\begin{operator}{Proves}{⊦}
\item.
\end{operator}

\begin{operator}{Half}{½}
\item.
\end{operator}

\begin{operator}{Delete}{⌫}
\item.
\end{operator}

\begin{operator}{Stop}{Ɂ}
\item.
\end{operator}

\begin{operator}{Primorial}{‼}
\item.
\end{operator}

\begin{operator}{Dagger}{†}
\item.
\end{operator}

\begin{operator}{Double Dagger}{‡}
\item.
\end{operator}

\begin{operator}{Bullet}{•}
\item.
\end{operator}

\begin{operator}{Begin Function}{‘}
    \op{}{FUNCTION_DEFINITION}: Returns an empty function definition.
\end{operator}

\begin{operator}{End Function}{’}
    \op{FUNCTION_DEFINITION}{function}: Compiles the definition of a function to
an actual callable function.
\end{operator}

\begin{operator}{Begin String}{“}
\item.
\end{operator}

\begin{operator}{End String}{”}
\item.
\end{operator}

\begin{operator}{Flatten}{⋯}
\item.
\end{operator}

\begin{operator}{Expand}{⋮}
\item.
\end{operator}

\begin{operator}{Ascending}{⋰}
    \op{list}{list}: Sorts the list in ascending order.
\end{operator}

\begin{operator}{Descending}{⋱}
    \op{list}{list}: Sorts the list in descending order.
\end{operator}

\begin{operator}{Graph}{⁜}
\item.
\end{operator}

\begin{operator}{One}{Ⅰ}
\item.
\end{operator}

\begin{operator}{Two}{Ⅱ}
\item.
\end{operator}

\begin{operator}{Three}{Ⅲ}
\item.
\end{operator}

\begin{operator}{Four}{Ⅳ}
\item.
\end{operator}

\begin{operator}{Five}{Ⅴ}
\item.
\end{operator}

\begin{operator}{Six}{Ⅵ}
\item.
\end{operator}

\begin{operator}{Seven}{Ⅶ}
\item.
\end{operator}

\begin{operator}{Eight}{Ⅷ}
\item.
\end{operator}

\begin{operator}{Nine}{Ⅸ}
\item.
\end{operator}

\begin{operator}{Ten}{Ⅹ}
\item.
\end{operator}

\begin{operator}{Eleven}{Ⅺ}
\item.
\end{operator}

\begin{operator}{Twelve}{Ⅻ}
\item.
\end{operator}

\begin{operator}{Fifty}{Ⅼ}
\item.
\end{operator}

\begin{operator}{Hundred}{Ⅽ}
\item.
\end{operator}

\begin{operator}{Five Hundred}{Ⅾ}
\item.
\end{operator}

\begin{operator}{Thousand}{Ⅿ}
\item.
\end{operator}

\begin{operator}{Alpha}{ɑ}
\item.
\end{operator}

\begin{operator}{Beta}{β}
\item.
\end{operator}

\begin{operator}{Five Thousand}{ↁ}
\item.
\end{operator}

\begin{operator}{Ten Thousand}{ↂ}
\item.
\end{operator}

\begin{operator}{Fifty Thousand}{ↇ}
\item.
\end{operator}

\begin{operator}{Hundred Thousand}{ↈ}
\item.
\end{operator}

\begin{operator}{Pop}{↤}
\item.
\end{operator}

\begin{operator}{Push}{⇤}
\item.
\end{operator}

\begin{operator}{Delta}{δ}
\item.
\end{operator}

\begin{operator}{Gamma}{Γ}
\item.
\end{operator}

\begin{operator}{Theta}{Θ}
\item.
\end{operator}

\begin{operator}{Lambda}{Λ}
\item.
\end{operator}

\begin{operator}{Xi}{Ξ}
\item.
\end{operator}

\begin{operator}{Phi}{Φ}
\item.
\end{operator}

\begin{operator}{Psi}{Ψ}
\item.
\end{operator}

\begin{operator}{Omega}{Ω}
\item.
\end{operator}

\begin{operator}{Complex}{ℂ}
\item.
\end{operator}

\begin{operator}{Quaternion}{ℍ}
\item.
\end{operator}

\begin{operator}{Natural}{ℕ}
\item.
\end{operator}

\begin{operator}{Prime}{ℙ}
\op{natural}{bool}: Check if the number is prime.
\end{operator}

\begin{operator}{Rational}{ℚ}
\item.
\end{operator}

\begin{operator}{Real}{ℝ}
\item.
\end{operator}

\begin{operator}{Integer}{ℤ}
\item.
\end{operator}

\begin{operator}{Pi}{π}
\item.
\end{operator}

\begin{operator}{Euler}{ℯ}
\item.
\end{operator}

\begin{operator}{Aleph}{א}
\item.
\end{operator}

\begin{operator}{Prescribe}{℞}
\item.
\end{operator}

\begin{operator}{Oplus}{⊕}
\item.
\end{operator}

\begin{operator}{Ominus}{⊖}
\item.
\end{operator}

\begin{operator}{Otimes}{⊗}
\item.
\end{operator}

\begin{operator}{Odivide}{⊘}
\item.
\end{operator}

\begin{operator}{Oequals}{⊜}
\item.
\end{operator}

\section{Examples}

\

\end{document}
